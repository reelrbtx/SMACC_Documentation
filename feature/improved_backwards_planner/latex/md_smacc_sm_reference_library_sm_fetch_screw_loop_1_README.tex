\subsection*{State Machine Diagram}



\subsection*{Description}

This example demonstrates the use of both Move\+It and Move\+Base within the same state machine, and cross orthogonal communication between the robot arm orthogonal and the perception orthogonal.~\newline


\href{https://reelrbtx.github.io/SMACC_Documentation/master/html/namespacesm__moveit.html}{\tt Doxygen Namespace \& Class Reference}

\subsection*{Build Instructions}

Before you build, make sure you\textquotesingle{}ve installed all the dependencies...


\begin{DoxyCode}
rosdep install --from-paths src --ignore-src -r -y 
\end{DoxyCode}


Then you build with either catkin build or catkin make...


\begin{DoxyCode}
catkin build
\end{DoxyCode}


\subsection*{Operating Instructions}

After you build, remember to source the proper devel folder...


\begin{DoxyCode}
source ~/catkin\_ws/devel/setup.bash
\end{DoxyCode}


And then run the launch file...


\begin{DoxyCode}
roslaunch sm\_fetch\_screw\_loop\_1 sm\_fetch\_screw\_loop\_1.launch
\end{DoxyCode}


\subsection*{Viewer Instructions}

If you have the S\+M\+A\+CC Viewer installed then type...


\begin{DoxyCode}
rosrun smacc\_viewer smacc\_viewer\_node.py
\end{DoxyCode}


If you don\textquotesingle{}t have the S\+M\+A\+CC Viewer installed, click \href{http://smacc.ninja/smacc-viewer/}{\tt here} for instructions. 