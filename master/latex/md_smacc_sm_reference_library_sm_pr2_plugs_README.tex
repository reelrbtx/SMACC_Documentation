\subsection*{State Machine Diagram}



\subsection*{Description}

Another simple, but complete state machine example that uses the \hyperlink{namespaceros__timer__client}{ros\+\_\+timer\+\_\+client} \& keyboard client to iterate through the states, which correspond to the days of the week. This example is another excellent starting point for new users state machine projects.~\newline


\href{https://reelrbtx.github.io/SMACC_Documentation/master/html/namespacesm__three__some.html}{\tt Doxygen Namespace \& Class Reference}

\subsection*{Build Instructions}

Before you build, make sure you\textquotesingle{}ve installed all the dependencies...


\begin{DoxyCode}
1 rosdep install --from-paths src --ignore-src -r -y 
\end{DoxyCode}


Then you build with either catkin build or catkin make...


\begin{DoxyCode}
1 catkin build
\end{DoxyCode}


\subsection*{Operating Instructions}

After you build, remember to source the proper devel folder...


\begin{DoxyCode}
1 source ~/catkin\_ws/devel/setup.bash
\end{DoxyCode}


And then run the launch file...


\begin{DoxyCode}
1 roslaunch sm\_pr2\_plugs sm\_pr2\_plugs.launch
\end{DoxyCode}


\subsection*{Viewer Instructions}

If you have the S\+M\+A\+CC Viewer installed then type...


\begin{DoxyCode}
1 rosrun smacc\_viewer smacc\_viewer\_node.py
\end{DoxyCode}


If you don\textquotesingle{}t have the S\+M\+A\+CC Viewer installed, click \href{http://smacc.ninja/smacc-viewer/}{\tt here} for instructions. 