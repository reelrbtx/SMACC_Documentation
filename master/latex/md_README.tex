\tabulinesep=1mm
\begin{longtabu} spread 0pt [c]{*3{|X[-1]}|}
\hline
\rowcolor{\tableheadbgcolor}{\bf R\+OS Distro }&{\bf Travis Build Status }&{\bf Documentation  }\\\cline{1-3}
\endfirsthead
\hline
\endfoot
\hline
\rowcolor{\tableheadbgcolor}{\bf R\+OS Distro }&{\bf Travis Build Status }&{\bf Documentation  }\\\cline{1-3}
\endhead
Indigo & &\href{https://reelrbtx.github.io/SMACC/indigo-devel/html/md_README.html}{\tt doxygen} \\\cline{1-3}
Kinetic & &\href{https://reelrbtx.github.io/SMACC/kinetic-devel/html/md_README.html}{\tt doxygen} \\\cline{1-3}
Melodic & &\href{https://reelrbtx.github.io/SMACC/melodic-devel/html/md_README.html}{\tt doxygen} \\\cline{1-3}
Master & &\href{https://reelrbtx.github.io/SMACC_Documentation/master/html/namespaces.html}{\tt doxygen} \\\cline{1-3}
\end{longtabu}


\subsection*{Docker Containers}

\href{https://hub.docker.com/r/pabloinigoblasco/smacc/}{\tt } \href{https://hub.docker.com/r/pabloinigoblasco/smacc/}{\tt } \href{https://registry.hub.docker.com/pabloinigoblasco/smacc/}{\tt }

\section*{S\+M\+A\+CC}

S\+M\+A\+CC is an event-\/driven, asynchronous, behavioral state machine library for real-\/time R\+OS (Robotic Operating System) applications written in C++, designed to allow programmers to build robot control applications for multicomponent robots, in an intuitive and systematic manner.

S\+M\+A\+CC is inspired by the \href{http://wiki.ros.org/smach}{\tt S\+M\+A\+CH R\+OS package} and it is built on top of \href{https://www.boost.org/doc/libs/1_53_0/libs/statechart/doc/index.html}{\tt Boost State\+Chart library}.

Probably the greatest strength of S\+M\+A\+CC is that it offers out-\/of-\/the-\/box reference state machines, (found in the folder \href{https://github.com/reelrbtx/SMACC/tree/master/smacc_sm_reference_library}{\tt sm\+\_\+reference\+\_\+library}) that you can use, test, hack, and customize to quickly get your application up and running, while also knowing that the library supports advanced functionalities that are practically universal among actual working robots.

\subsection*{Features}


\begin{DoxyItemize}
\item $\ast$$\ast$$\ast$\+Powered by R\+OS\+:$\ast$$\ast$$\ast$ S\+M\+A\+CC has been developed specifically to work with R\+OS. It is a c++ ros package that can be imported from any end-\/user application package.
\item $\ast$$\ast$$\ast$\+Written in C++\+:$\ast$$\ast$$\ast$ Until now, R\+OS has lacked a library to develop task-\/level state machines in C++. Although libraries have been developed in scripting languages such as python, these are unsuitable for real-\/world industrial evironments where real-\/time requirements are demanded.
\item $\ast$$\ast$$\ast$\+Static State Machine Checking\+:$\ast$$\ast$$\ast$ S\+M\+A\+CC inherits this from the Boost Statechart library which helps the developer to check the consistency of the state machine at compile time (instead of runtime).
\item $\ast$$\ast$$\ast$\+Component based architecture\+:$\ast$$\ast$$\ast$ S\+M\+A\+CC has built-\/in funcionality provided inside S\+M\+A\+CC Components that can be dynamically imported at runtime and stored in the local machine. The states only access those components they are concerned with. This enables the S\+M\+A\+CC based application to extend or improve the runtime behavior of the system.
\end{DoxyItemize}

\subsection*{S\+M\+A\+CC applications}

From it\textquotesingle{}s inception, S\+M\+A\+CC was written to support the programming of multi-\/component, complex robots. If your project involves small, solar-\/powered insect robots, that simply navigate towards a light source, then S\+M\+A\+CC might not be the right choice for you. But if you are trying to program a robot with a mobile base, a robotic arm, a gripper, two lidar sensors, a gps transceiver and an imu, then you\textquotesingle{}ve come to the right place.

\subsection*{R\+OS Integration}


\begin{DoxyItemize}
\item $\ast$$\ast$$\ast$\+Intensive use of R\+OS Action$\ast$$\ast$$\ast$. S\+M\+A\+CC translates Action server events (Result callbacks, Feedback callbacks, etc.) into statechart events. To learn more about this, check the sections Shared Resources and S\+M\+A\+CC Architecture.
\item $\ast$$\ast$$\ast$\+Powerful access to R\+OS Parameters$\ast$$\ast$$\ast$. Each S\+M\+A\+CC state automatically creates a ros\+::\+Node\+Handle automatically named according to the S\+M\+A\+CC state hierarchy (see more in section Usage Examples -\/ Ros parameters)
\item $\ast$$\ast$$\ast$\+R\+OS Navigation built-\/in funcionality$\ast$$\ast$$\ast$. S\+M\+A\+CC extends the R\+OS navigation stack in a high level way. It provides specialized navigation planners (for the R\+OS Navigation Stack) that navigate only using pure spinning motions and straight motions. Implements some mechanism to perform motions recording the path and undoing them later. These can be very useful in some industrial applications where the knowledge or certainty on the environment is higher (ros planners are focused on cluttered and dynamic environments).
\end{DoxyItemize}

\subsection*{Future Work}


\begin{DoxyItemize}
\item undoing paths chunks by state (store the different chunks of the path according to its state in a stack)
\item code generation based on uml diagrams
\item improving backwards planners for non linear paths
\end{DoxyItemize}

\subsection*{Development methodology}

S\+M\+A\+CC also defines a development methodology where State Machine nodes only contain the task-\/level logic, that is, the high level behavior of the robot system in some specific application.

S\+M\+A\+CC applications have low level coupling with other software components of the robot system. S\+M\+A\+CC code is recomended to interact with the rest of components the robot system via R\+OS Action Servers and {\bfseries Smacc Action Plugins}.

The proposed methdology split the states into 2 or more statechart orthogonal lines that comunicate to each other via events. The orthogonal line 0 is tipically for the mobile robot navigation. The second orthogonal line and ahead are used for tools (manipulators, grippers or other custom tools).

 

\subsection*{Internal Architecture}

S\+M\+A\+CC State Machines are boost\+::statechart Asynchronous\+State\+Machines that can work in a multi-\/threaded application. In S\+M\+A\+CC State Machines are two main components that work concurrently in two different threads\+:


\begin{DoxyItemize}
\item $\ast$$\ast$$\ast$\+Signal Detector$\ast$$\ast$$\ast$. It is able to handle the action client components communication with action servers and translate them to statechart events
\item $\ast$$\ast$$\ast$\+State Machine$\ast$$\ast$$\ast$. It is the end-\/user code of the state machine itself.
\end{DoxyItemize}

 

\section*{Tutorial}

S\+M\+A\+CC states inherits from boost\+::statechart\+:State so that you can learn the full potential of S\+M\+A\+CC states also diving in the statechart documentation. However, the following examples briefly show how you create define S\+M\+A\+CC states and how you would usually use them.

\subsection*{Getting Started}

The easiest way to get started is by selecting one of the state machines in our reference library, and then hacking it to meet your needs.

Each state machine in the reference library comes with it\textquotesingle{}s own R\+E\+A\+D\+M\+E.\+md file, which contains the appropriate operating instructions, so that all you have to do is simply copy \& paste some commands into your terminal.

\subsection*{Anatomy of a simple S\+M\+A\+CC State}

For the previous state machine, this would be the initial S\+M\+A\+CC State. It also follows the Curiously recurrent template pattern. However, for Smacc states, the second template parameters is the so called \char`\"{}\+Context\char`\"{}, for this simple case, the context is the State\+Machine type itself. However, that could also be other State (in a nexted-\/substate case) or an orthogonal line.


\begin{DoxyCode}
\textcolor{keyword}{struct }ToolSimpleState
    : \hyperlink{classSmaccState}{SmaccState}<ToolSimpleState, SimpleStateMachine>
\{
\textcolor{keyword}{public}:

  \textcolor{keyword}{using} SmaccState::SmaccState;
  \textcolor{keywordtype}{void} onEntry()
  \{
    ROS\_INFO(\textcolor{stringliteral}{"Entering ToolSimpleState"});
  \}
\};

\textcolor{keywordtype}{int} \hyperlink{odom__tracker__node_8cpp_a3c04138a5bfe5d72780bb7e82a18e627}{main}(\textcolor{keywordtype}{int} argc, \textcolor{keywordtype}{char} **argv) \{
  \textcolor{comment}{// initialize the ros node}
  ros::init(argc, argv, \textcolor{stringliteral}{"example1"});
  ros::NodeHandle nh;

  smacc::run<SimpleStateMachine>();
\}
\end{DoxyCode}
 According to the U\+ML statchart standard, things happens essencially when the system enters in the state, when the system exits the state and when some event is triggered. The two first ones are shown in this example. The c++ Constructor code is the place you have to write your \char`\"{}entry code\char`\"{}, the destructor is the place you have to write your \char`\"{}exit code\char`\"{}. The constructor parameter (my\+\_\+context) is a reference to the context object (in this case the state machine). This kind of constructor may be verebosy, but is required to implement the rest of S\+M\+A\+CC tasks and always follows the same pattern.

\subsection*{Simple State Transition on Action Result Event}

According to the U\+ML state machines standard, transitions between states happen on events. In S\+M\+A\+CC events can be implemented by the user or happen when Action Results callbacks and Action Feedback callbacks happen. In the following example we extend the previous example to transit to another state \textquotesingle{}Execute\+Tool\+State\textquotesingle{} when the move\+\_\+base action sever returns a Result.

 

The following would be the code to implement the diagram shown above.


\begin{DoxyCode}
\textcolor{keyword}{struct }Navigate : \hyperlink{classSmaccState}{SmaccState}<Navigate, SimpleStateMachine>
\{
\textcolor{keyword}{public}:

  \textcolor{comment}{// With this line we specify that we are going to react to any EvActionResult event}
  \textcolor{comment}{// generated by SMACC when the action server provides a response to our request}
  \textcolor{keyword}{typedef} mpl::list<sc::transition<EvActionResult<SmaccMoveBaseActionClient::Result>, ExecuteToolState>> 
      reactions;

  \textcolor{keyword}{using} SmaccState::SmaccState;
  \textcolor{keywordtype}{void} onEntry()
  \{
   [...]
  \}
\};

\textcolor{keyword}{struct }ExecuteToolState : \hyperlink{classSmaccState}{SmaccState}<ExecuteToolState, SimpleStateMachine>
\{
    \textcolor{keyword}{using} SmaccState::SmaccState;
    \textcolor{keywordtype}{void} onEntry()
    \{
    \}
\};
\end{DoxyCode}


\subsection*{Adding R\+OS Parameters to Smacc States}

The S\+M\+A\+CC states can be configured from the ros parameter server based on their hierarchy and their class name. It is responsability of the user not to have two different state names at the same level (even if the namespace is distinct since the namespace is trimmed for parameters)

For example, imagine a State\+Machine to move the mobile robot initially to some initial position, and then moving it to some other position relative to the initial position. You could put the navigation parameters in a ros configuration yaml file like this (and avoid hardcoding)\+:


\begin{DoxyCode}
1 MyStateMachine:
2     State1: #Go to some initial position
3         NavigationOrthogonalLine:
4             Navigate:
5                 start\_position\_x: 3
6                 start\_position\_y: 0
7     State2: #Go to some initial position
8         NavigationOrthogonalLine:
9             Navigate:
10                 initial\_orientation\_index: 0 # the initial index of the linear motion (factor of
       angle\_increment\_degrees)
11                 angle\_increment\_degree: 90    # the increment of angle between to linear motions
12                 linear\_trajectories\_count: 2  # the number of linear trajectories of the radial 
\end{DoxyCode}


Then, the c++ code for the State My\+State\+Machine/\+State1/\+Navigation\+Orthogonal\+Line/\+Navigate could contain the following parameter reading funcionality\+:


\begin{DoxyCode}
\textcolor{keyword}{struct }Navigate : \hyperlink{classSmaccState}{SmaccState}<Navigate, NavigationOrthogonalLine> 
\{
\textcolor{keyword}{public}:
  \textcolor{keyword}{using} SmaccState::SmaccState;
  \textcolor{keywordtype}{void} onEntry()
  \{
      geometry\_msgs::Point p;
      param(\textcolor{stringliteral}{"start\_position\_x"}, p.x, 0);
      param(\textcolor{stringliteral}{"start\_position\_y"}, p.y, 0);
  \}
\}
\end{DoxyCode}


The param template method reads from the parameters server delegating to the method defined ros\+::\+Node\+Handle handle does but already located at the exact point in the parameter name hierarchy associated to this state. S\+M\+A\+CC is also able have methods get\+Param and set\+Param that are delegated to ros\+::\+Node\+Handle in the same way. 