\tabulinesep=1mm
\begin{longtabu} spread 0pt [c]{*{3}{|X[-1]}|}
\hline
\rowcolor{\tableheadbgcolor}\textbf{ R\+OS Distro }&\textbf{ Travis Build Status }&\textbf{ Documentation  }\\\cline{1-3}
\endfirsthead
\hline
\endfoot
\hline
\rowcolor{\tableheadbgcolor}\textbf{ R\+OS Distro }&\textbf{ Travis Build Status }&\textbf{ Documentation  }\\\cline{1-3}
\endhead
Kinetic & &\href{https://reelrbtx.github.io/SMACC_Documentation/kinetic-devel/html/namespaces.html}{\tt doxygen} \\\cline{1-3}
Melodic & &\href{https://reelrbtx.github.io/SMACC_Documentation/melodic-devel/html/namespaces.html}{\tt doxygen} \\\cline{1-3}
Master & &\href{https://reelrbtx.github.io/SMACC_Documentation/master/html/namespaces.html}{\tt doxygen} \\\cline{1-3}
\end{longtabu}


\subsection*{Docker Containers}

\href{https://hub.docker.com/r/pabloinigoblasco/smacc/}{\tt } \href{https://hub.docker.com/r/pabloinigoblasco/smacc/}{\tt } \href{https://registry.hub.docker.com/pabloinigoblasco/smacc/}{\tt }

\section*{S\+M\+A\+CC}

S\+M\+A\+CC is an event-\/driven, asynchronous, behavioral state machine library for real-\/time R\+OS (Robotic Operating System) applications written in C++, designed to allow programmers to build robot control applications for multicomponent robots, in an intuitive and systematic manner.

S\+M\+A\+CC was inspired by Harel\textquotesingle{}s statecharts and the \href{http://wiki.ros.org/smach}{\tt S\+M\+A\+CH R\+OS package}. S\+M\+A\+CC is built on top of the \href{https://www.boost.org/doc/libs/1_53_0/libs/statechart/doc/index.html}{\tt Boost State\+Chart library}.

\subsection*{Features}


\begin{DoxyItemize}
\item {\itshape {\bfseries Powered by R\+OS\+:}} S\+M\+A\+CC has been developed specifically to work with R\+OS. It supports R\+OS topics, services and actions, right out of the box.
\item {\itshape {\bfseries Written in C++\+:}} Until now, R\+OS has lacked a library to develop task-\/level behavioral state machines in C++. Although libraries have been developed in scripting languages such as python, these are unsuitable for real-\/world industrial evironments where real-\/time requirements are demanded.
\item {\itshape {\bfseries Orthogonals\+:}} Originally conceived by David Harel in 1987, orthogonality is absolutely crucial to developing state machines for complex robotic systems. This is because complex robots are always a collection of hardware devices which require communication protocols, start-\/up determinism, etc. With orthogonals, it is an intuitive and relatively straight forward exercise (at least conceptully;) to code a state machine for a robot comprising a mobile base, a robotic arm, a gripper, two lidar sensors, a gps transceiver and an imu, for instance.
\item {\itshape {\bfseries Static State Machine Checking\+:}} One of the features that S\+M\+A\+CC inherits from Boost Statechart is that you get compile time validation checking. This benefits developers in that the amount of runtime testing necessary to ship quality software that is both stable and safe is dramatically reduced. Our philosophy is \char`\"{}\+Wherever possible, let the compiler do it\char`\"{}.
\item {\itshape {\bfseries State Machine Reference Library\+:}} With a constantly growing library of out-\/of-\/the-\/box reference state machines, (found in the folder \href{https://github.com/reelrbtx/SMACC/tree/master/smacc_sm_reference_library}{\tt sm\+\_\+reference\+\_\+library}) guaranteed to compile and run, you can jumpstart your development efforts by choosing a reference machine that is closest to your needs, and then customize and extend to meet the specific requirements of your robotic application. All the while knowing that the library supports advanced functionalities that are practically universal among actual working robots.
\item {\itshape {\bfseries S\+M\+A\+CC Client Library\+:}} S\+M\+A\+CC also features a constantly growing library of \href{https://github.com/reelrbtx/SMACC/tree/master/smacc_client_library}{\tt clients} that support R\+OS Action Servers, Service Servers and other nodes right out-\/of-\/the box. The clients within the S\+M\+A\+CC Client library have been built utilizing a component based architecture that allows for developer to build powerful clients of their own. Current clients of note include Move\+BaseZ, a full featured Action Client built to integrate with the R\+OS Navigation stack, the ros\+\_\+timer\+\_\+client, the multi\+\_\+role\+\_\+sensor\+\_\+client, and a keyboard\+\_\+client used extensively for state machine drafting \& debugging.
\begin{DoxyItemize}
\item {\itshape {\bfseries Extensive Documentation\+:}} Although many R\+OS users are familiar with doxygen, our development team has spent a lot of time researching the more advanced features of doxygen such as uml style class diagrams and call graphs, and we\textquotesingle{}ve used them to document the S\+M\+A\+CC library. Have a look to \href{https://reelrbtx.github.io/SMACC_Documentation/master/html/namespaces.html}{\tt our doxygen sites}) and we think you\textquotesingle{}ll be blown away at what Doxygen looks like when \href{https://reelrbtx.github.io/SMACC_Documentation/master/html/classsmacc_1_1ISmaccStateMachine.html}{\tt it\textquotesingle{}s done right}) and it becomes a powerful tool to research a codebase.
\item {\itshape {\bfseries S\+M\+A\+CC Viewer\+:}} The S\+M\+A\+CC library works out of the box with the S\+M\+A\+CC Viewer. This allows developers to visualize and runtime debug the state machines they are working on. The S\+M\+A\+CC Viewer is closed source, but is free and can be \href{http://smacc.ninja/smacc-viewer/}{\tt installed} via apt-\/get. To view the S\+M\+A\+CC Viewer in action, click \href{https://www.youtube.com/watch?v=WVt4M_teA5I}{\tt here} and \href{https://www.youtube.com/watch?v=fdy37WvC4FQ}{\tt here}. Be sure to set the youtube video to 720p HD.
\end{DoxyItemize}
\end{DoxyItemize}

\subsection*{S\+M\+A\+CC applications}

From it\textquotesingle{}s inception, S\+M\+A\+CC was written to support the programming of multi-\/component, complex robots. If your project involves small, solar-\/powered insect robots, that simply navigate towards a light source, then S\+M\+A\+CC might not be the right choice for you. But if you are trying to program a robot with a mobile base, a robotic arm, a gripper, two lidar sensors, a gps transceiver and an imu, then you\textquotesingle{}ve come to the right place.

 

\subsection*{Getting Started}

The easiest way to get started is by selecting one of the state machines in our \href{https://github.com/reelrbtx/SMACC/tree/master/smacc_sm_reference_library}{\tt reference library}, and then hacking it to meet your needs.

Each state machine in the reference library comes with it\textquotesingle{}s own R\+E\+A\+D\+M\+E.\+md file, which contains the appropriate operating instructions, so that all you have to do is simply copy \& paste some commands into your terminal.


\begin{DoxyItemize}
\item If you are looking for a minimal example, we recommend \href{https://github.com/reelrbtx/SMACC/tree/master/smacc_sm_reference_library/sm_atomic}{\tt sm\+\_\+atomic}.
\item If you are looking for a slightly more complicated, but still very simple example, try \href{https://github.com/reelrbtx/SMACC/tree/master/smacc_sm_reference_library/sm_calendar_week}{\tt sm\+\_\+calendar\+\_\+week}.
\item If you are looking for a minimal example but with a looping superstate, try \href{https://github.com/reelrbtx/SMACC/tree/master/smacc_sm_reference_library/sm_three_some}{\tt sm\+\_\+three\+\_\+some}.
\item If you want to get started with the R\+OS Navigation stack right away, try \href{https://github.com/reelrbtx/SMACC/tree/master/smacc_sm_reference_library/sm_dance_bot}{\tt sm\+\_\+dance\+\_\+bot}.
\item If you want to get started with R\+OS Navigation and exploring the orthogonal read-\/write cycle, then try \href{https://github.com/reelrbtx/SMACC/tree/master/smacc_sm_reference_library/sm_dance_bot_strikes_back}{\tt sm\+\_\+dance\+\_\+bot\+\_\+strikes\+\_\+back}.
\end{DoxyItemize}

Operating instructions can be found in each reference state machines readme file. Happy Coding.

\subsection*{Support}

If you are interested in getting involved or need a little support, feel free to contact us by emailing \href{mailto:brett@reelrobotix.com}{\tt brett@reelrobotix.\+com} 