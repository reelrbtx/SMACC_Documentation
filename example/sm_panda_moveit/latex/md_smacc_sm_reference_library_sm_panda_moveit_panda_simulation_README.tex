

This package was written for R\+OS melodic running under Ubuntu 18.\+04. Run the following commands to make sure that all additional packages are installed\+:


\begin{DoxyCode}
mkdir -p catkin\_ws/src
cd catkin\_ws/src
git clone https://github.com/erdalpekel/panda\_simulation.git
git clone https://github.com/erdalpekel/panda\_moveit\_config.git
git clone --branch simulation https://github.com/erdalpekel/franka\_ros.git
cd ..
sudo apt-get install libboost-filesystem-dev
rosdep install --from-paths src --ignore-src -y --skip-keys libfranka
cd ..
\end{DoxyCode}
 It is also important that you build the {\itshape libfranka} library from source and pass its directory to {\itshape catkin\+\_\+make} when building this R\+OS package as described in \href{https://frankaemika.github.io/docs/installation.html#building-from-source}{\tt this tutorial}.

Currently it includes a controller parameter config file and a launch file to launch the \href{http://gazebosim.org}{\tt Gazebo} simulation environment and the Panda robot from F\+R\+A\+N\+KA E\+M\+I\+KA in it with the necessary controllers.

Build the catkin workspace and run the simulation\+: 
\begin{DoxyCode}
catkin\_make -j4 -DCMAKE\_BUILD\_TYPE=Release -DFranka\_DIR:PATH=/path/to/libfranka/build
source devel/setup.bash
roslaunch panda\_simulation simulation.launch
\end{DoxyCode}


Depending on your operating systems language you might need to export the numeric type so that rviz can read the floating point numbers in the robot model correctly\+:


\begin{DoxyCode}
export LC\_NUMERIC="en\_US.UTF-8"
\end{DoxyCode}
 Otherwise, the robot will appear in rviz in a collapsed state.

You can see the full explanation in my \href{https://erdalpekel.de/?p=55}{\tt blog post}.

\subsection*{Changelog\+:}

\href{https://erdalpekel.de/?p=123}{\tt {\itshape Move\+It!} constraint-\/aware planning}

This repository was extended with a R\+OS node that communicates with the {\itshape Move\+It!} Planning Scene A\+PI. It makes sure that the motion planning pipeline avoids collision objects in the environment specified by the user in a separate directory ({\ttfamily $\sim$/.\hyperlink{namespacepanda__simulation}{panda\+\_\+simulation}}) as {\itshape json} files.

\href{https://erdalpekel.de/?p=123}{\tt Publishing a box at Panda\textquotesingle{}s hand in {\itshape Gazebo}}

This repository was extended with a node that publishes a simple box object in the {\itshape Gazebo} simulation at the hand of the robot. The position of this box will get updated as soon as the robot moves.

\href{https://erdalpekel.de/?p=123}{\tt Visual Studio Code Remote Docker}

I have added configuration files and additional setup scripts for developing and using this R\+OS package within a {\itshape Docker} container. Currently user interfaces for Gazebo and R\+Viz are not supported.

\href{https://erdalpekel.de/?p=285}{\tt Position based trajectory execution}

The joint specifications in Gazebo were changed from an effort interface to position based interface. Furthermore, the P\+ID controller was substituted with the simple gazebo internal position based control mechanism for a more stable movement profile of the robot. A custom joint position based controller was implemented in order to set the initial joint states of the robot to a valid configuration.

\href{https://erdalpekel.de/?p=314}{\tt Automatic robot state initialization}

A separate R\+OS node was implemented that starts a custom joint position controller and initializes the robot with a specific configuration. It switches back to the default controllers after the robot reaches the desired state.

 